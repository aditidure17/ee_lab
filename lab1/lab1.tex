\documentclass[a4paper,12pt]{article}

% Packages
\usepackage{geometry}
\usepackage{graphicx}
\usepackage{amsmath}
\usepackage{caption}
\usepackage{subcaption}
\usepackage{listings}
\usepackage{float}
\usepackage{hyperref}
\usepackage{circuitikz}

% Page setup
\geometry{margin=1in}

\begin{document}

\begin{figure}
   \centering
   \includegraphics[width=0.8\textwidth]{./images.png}
\end{figure}


\begin{center}
    Year : 2023-24 \\
    Subject Code : EE1200 \\
    Branch : Electrical Engineering \\
    Name : Aditi Dure 
\end{center}

\newpage
\title{Verification of KCL and KVL}
\date{}
\maketitle

\section{Aim}
To verify Kirchhoff’s Voltage Law (KVL) and Kirchhoff’s Current Law (KCL) in the given circuit

\section{Circuit Diagram}
\subsection{KCL: }
\begin{center}
  \begin{circuitikz}
      \draw (0,0)
      to[V, v=$V$] (0,2);
      
      \draw (0,2)
      -- ++(1,0)
      to[R, l=$R_1$] ++(0,-2)
      -- (0,0);

      \draw (1,2)
      -- ++(1,0)
      to[R, l=$R_2$] ++(0,-2)
      -- (1,0);
      
      \draw (2,2)
      -- ++(1,0)
      to[R, l=$R_3$] ++(0,-2)
      -- (2,0);

      % Mark a node at the junction
      \node[circle,fill,inner sep=2pt,label={above:$A$}] at (1,2) {};
  \end{circuitikz}
\end{center}

where 
\begin{align}
\text{V} &= 5 \text{V} \\
R_1 &= 100 \Omega \\
R_2 &= 220 \Omega \\
R_2 &= 180 \Omega
\end{align}

\subsection{KVL: }
\begin{center}
\begin{circuitikz}
    \draw (0,0)
    to[V, v=$V$] (0,2)
    to[R, l=$R_1$] (2,2)
    to[R, l=$R_2$] (4,2)
    to[R, l=$R_3$] (6,2)
    -- (6,0)
    -- (0,0);
\end{circuitikz}
\end{center}

where 
\begin{align}
\text{V} &= 5 \text{V} \\
R_1 &= 100 \Omega \\
R_2 &= 220 \Omega \\
R_2 &= 180 \Omega
\end{align}


\section{Theory}
KCL: \\ 
At any node in an electrical circuit, the net current is zero. \\
KVL: \\
In closed path of an electrical circuit the net potential around the closed path is zero. \\

\section{Theoretical calculations: } 
\subsection{For verifying KCL: }
Applying  KCL at node A: \\
To find I,
\begin{align}
    I &= I_1 + I_2 + I_3 \\
    R_{\text{eq}} &= \left( \frac{1}{100} + \frac{1}{180} + \frac{1}{220} \right)^{-1} \\
    R_{\text{eq}} &= 49.74 \, \Omega \\[1em]
    I &= \frac{V}{R_{\text{eq}}} \\
    I &= \frac{5}{49.74} \\
    I &= 0.1 \, \text{A}
\end{align}
To find \(I_1\), \(I_2\), \(I_3\) using current division rule:
\begin{align}
    I_1 &= \frac{100 \cdot \text{I}}{100 + 220 + 180} = 0.02 \, \text{A} \\
    I_2 &= \frac{220 \cdot \text{I}}{100 + 220 + 180} = 0.044 \, \text{A} \\
    I_3 &= \frac{180 \cdot \text{I}}{100 + 220 + 180} = 0.036 \, \text{A}
\end{align}

\subsection{For verifying KVL: }
{Assume current's direction to be clockwise} \\
Applying KVL in loop: \\
Assume the current to be I, \\
\begin{align}
-V + I \cdot R_1 + I \cdot R_2 + I \cdot R_3 &= 0 \\
-5 + I \cdot 100 + I \cdot 220 + I \cdot 180 &= 0
\end{align}
On solving, we get
\begin{align}
I &= 0.01 \, \text{A}
\end{align}
so,

\begin{align}
    V_1 &= I \cdot R_1 \nonumber = 0.01 \, \text{A} \cdot 100 \, \Omega \nonumber = 1 \, \text{V} \\
    V_2 &= I \cdot R_2 \nonumber = 0.01 \, \text{A} \cdot 220 \, \Omega \nonumber = 2.2 \, \text{V} \\
    V_1 &= I \cdot R_3 \nonumber = 0.01 \, \text{A} \cdot 180 \, \Omega \nonumber = 1.8 \, \text{V}
\end{align}


\section{Observation}
\subsection{For KCL: }
\begin{table}[h]
  \centering
  \begin{tabular}{|c|c|c|}
    \hline
    \textbf{} & \textbf{theoretical} & \textbf{practical} \\
    \hline
    I(A) &  $0.1\text{A}$ &  $0.09\text{A}$ \\
    $I_1$(A) &  $0.02\text{A}$ &  $0.019\text{A}$ \\
    $I_2$(A) &  $0.044\text{A}$ &  $0.045\text{A}$ \\
    $I_3$(A) &  $0.036\text{A}$ &  $0.037\text{A}$ \\
    \hline
  \end{tabular}
\end{table}
\subsection{For KVL: }
\begin{table}[h]
  \centering
  \begin{tabular}{|c|c|c|}
    \hline
    \textbf{} & \textbf{theoretical} & \textbf{practical} \\
    \hline
    $V_1$(V) &  $1\text{V}$ &  $1.2\text{V}$ \\
    $V_2$(V) &  $2.2\text{V}$ &  $2.1\text{V}$ \\
    $V_3$(V) &  $1.8\text{V}$ &  $1.7\text{V}$ \\
    V(V) &  $0.1\text{V}$ &  $0.1\text{V}$ \\
    \hline
  \end{tabular}
\end{table}

\section{Result}
For KCL, total input current is equal to total output current at node A.  \\
For KVL, the net potential in a loop is zero


\end{document}

